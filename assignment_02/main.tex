%& -job-name=Assignment2
\input{../documents-common/preamble.tex}
\begin{document}
\maketitle[Assignment 2]{Analysis of Numerical Methods}
\section{}
\begin{align}
\inp{u}{u_t} + \inp{u}{Au_x} = 0\nn
\to \frac{1}{2}\drv[t]{}\norm{u}^2 &= -\int_0^1 u^TAu_x\d{x}\nn
&= -\int_0^1 u_1(x)u_{2,x}(x)\d{x}-\int_0^1 u_2(x)u_{1,x}(x)\d{x}\nn
&= -u_1(x)u_2(x)\bigg|_0^1+\int_0^1 u_{1,x}(x)u_2(x)\d{x}-\int_0^1 u_2(x)u_{1,x}(x)\d{x}\nn
&= -u_1(x)u_2(x)\bigg|_0^1\nn
&= 0
\end{align}
Therefore the square of the norm of $u$ is constant,
 and given by the square norm in the initial condition. %order of square?
\begin{align}
 \norm{u(\cdot,t)}^2 = \norm{u(\cdot, 0)}^2 &= \norm{f(\cdot)}^2\nn
 \to \norm{u(\cdot,t)} &= \norm{f(\cdot)}
\end{align}

\section{}
Denote the semi-discrete version of $u$ by $v$, using the SBP operator $D$ for the first derivative we can express it as
\begin{align}
v_t = A\otimes Dv  = A\otimes H^{-1}(Q+R)v
\end{align}
And denote the discrete-energy method inner-product as
\begin{align}
\inp{x}{y}_H = x^T\left(H\otimes I\right)y = x^T\hat H y,
\end{align}.
Where $I$ is the $2\times 2$ identity matrix.
%TODO: text
\begin{align}
\inp{v}{v_t} + \inp{v_t}{v} &= \inp{v}{A \otimes Dv} + \inp{A \otimes Dv}{v}\nn
\drv[t]{}\norm{v}_H^2 &= v^THAH^{-1}(Q+R)v + v^T(Q+R)^TH^{-1}AH v\nn
&= v^T\left(HAH^{-1}(Q+R)+(Q^T+R^T)H^{-1}AH\right)v
\end{align}

\end{document}
