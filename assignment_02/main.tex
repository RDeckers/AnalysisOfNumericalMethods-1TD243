%& -jobname=Assignment2
\input{../documents-common/preamble.tex}
\begin{document}
\maketitle[Assignment 2]{Analysis of Numerical Methods}
\section{}
\begin{align}
\inp{u}{u_t} + \inp{u}{Au_x} = 0\nn
\to \frac{1}{2}\drv[t]{}\norm{u}^2 &= -\int_0^1 u^TAu_x\d{x}\nn
&= -\int_0^1 u_1(x)u_{2,x}(x)\d{x}-\int_0^1 u_2(x)u_{1,x}(x)\d{x}\nn
&= -u_1(x)u_2(x)\bigg|_0^1+\int_0^1 u_{1,x}(x)u_2(x)\d{x}-\int_0^1 u_2(x)u_{1,x}(x)\d{x}\nn
&= -u_1(x)u_2(x)\bigg|_0^1\nn
&= u_1(0)u_2(0) -u_1(1)u_2(1)
&= 0
\end{align}
Therefore the square of the norm of $u$ is constant,
 and given by the square norm in the initial condition. %order of square?
\begin{align}
 \norm{u(\cdot,t)}^2 = \norm{u(\cdot, 0)}^2 &= \norm{f(\cdot)}^2\nn
 \to \norm{u(\cdot,t)} &= \norm{f(\cdot)}
\end{align}

\section{}
Denote the semi-discrete version of $u$ by $v$, using the SBP operator $D$ for the first derivative we can express it as
\begin{align}
v_t = -A\otimes Dv  = -A\otimes H^{-1}(Q+R)v
\end{align}
And denote the discrete-energy method inner-product as
\begin{align}
\inp{x}{y}_H = x^T\left(I\otimes H\right)y = x^T\hat H y,
\end{align}.
Where $I$ is the $2\times 2$ identity matrix.
%TODO: text
Multiplying our equation by $v^T\hat H$ from the left gives us
\begin{align}
\inp{v}{v_t}_{\hat H} &= \inp{v}{-A\otimes D v}_{\hat H}\nn{}
&= -v^T\hat HA\otimes Dv\nn
&= -v^T(I\otimes H)(A\otimes D)v\nn
&= -v^T(A\otimes HD)v\nn
&= -v^T(A\otimes (Q+R))v
\end{align}
and the transpose gives us
\begin{align}
  \inp{v_t}{v}_{\hat H} &= \inp{-A\otimes D v}{v}_{\hat H}\nn{}
  &= -\left(A\otimes Dv)^T\hat H v\nn
  &= -v^T(A\otimes D)^T(I\otimes H)v\nn
  &= -v^T\left(A^TI\otimes D^TH\right)v\nn
  &= -v^T\left(A\otimes (Q+R)^TH^{-1}H\right)v
\end{align}

Summing these two expressions gives us the discrete energy estimate:
\begin{align}
\inp{v_t}{v}_{\hat H} + \inp{v}{v_t}_{\hat H} = \drv[t]{} \norm{v}^2_{\hat H} &= -v^T\left(A\otimes(Q+Q^T+R+R^T)\right)v\nn
&= -v^T\left(A\otimes \left(R+R^T\right)\right)v\nn  %TODO: Check this
&= 2\left(v^1_0v^2_0 - v^1_Nv^2_N\right)
\end{align}
We now add the SAT terms
\begin{align}
  \alpha\otimes H^{-1}e_0\left(e^1\otimes e^T_0\right)v = \alpha\otimes H^{-1}e_0v_0^1\\
  \beta\otimes H^{-1}e_N\left(e^2\otimes e^T_N\right)v = \beta\otimes H^{-1}e_Nv_N^2
\end{align}
where
\begin{align}
  e^1 = \left(1, 0\right)\nn
  e^2 = \left(0, 1\right)
\end{align}
and $\alpha$ and $\beta$ are 2D penalty row-vectors.
Taking the discrete inner-product w.r.t. $\hat H$:
\begin{align}
  v^T\left(I_2\otimes H\right)\alpha H^{-1}e_0v_0^1 = \alpha_0\left(v_0^1\right)^2 + \alpha_1\left(v_0^1v_0^2\right)\\
  v^T\left(I_2\otimes H\right)\beta H^{-1}e_Nv_N^2 = \beta_1\left(v_N^2\right)^2 + \beta_0\left(v_N^1v_N^2\right)
\end{align}
and the transpose
\begin{align}
\left[\alpha\otimes H^{-1}e_0v_0^1\right]^T\left(I_2\otimes H\right)v &= v_0^1e_0^T\left(\alpha\otimes H^{-1}\right)^T\left(I_2\otimes H\right)v\nn
&= v_0^1e_0^T\left(\alpha^T\otimes I_N\right)v\nn
&= v_0^1\left(\alpha_0v_0^1+\alpha_1v_0^2\right)\nn
&= \alpha_0\left(v_0^1\right)^2+\alpha_1\left(v_0^1v_0^2\right)\\
\left[\beta\otimes H^{-1}e_Nv_N^2\right]^T\left(I_2\otimes H\right)v &= \beta_1\left(v_N^2\right)^2 + \beta_0\left(v_N^1v_N^2\right)
\end{align}
\par Adding everything together we end up with our SBP-SAT energy estimate:
\begin{align}
  \frac{1}{2}\drv[t]{} \norm{v}^2_{\hat H} &=  v^1_0v^2_0 - v^1_Nv^2_N +   \beta_1\left(v_N^2\right)^2 + \beta_0\left(v_N^1v_N^2\right) + \alpha_0\left(v_0^1\right)^2 + \alpha_1\left(v_0^1v_0^2\right)
\end{align}
Which leads to a stability condition of
\begin{align}
  \beta_0 = 1,\;\alpha_1 = -1,\nn
  \beta_1 \leq 0,\;\alpha_0 \leq 0
\end{align}
Setting $\beta_1 = \alpha_0 = 0$ guarentees we have energy conservation, similar to what we found in the exact case.
\section{}
We do not obtain the expected convergence rate, we would expect to see something between the method order and the method order$/2 +1$
but we find an order of roughly $0.5$ for both methods. This is probably due to a software bug but we have been unable to trace it down in time.
We do observe energy conservation as expected.
\begin{figure}
  \centering
  \caption{Convergence graph, gray = $\O(h^{0.5})$, orange = `sixth-order', blue = `fourth-order'.}
 \includegraphics[width=0.75\textwidth]{plots/convergence}
\end{figure}
\begin{figure}
\centering
\caption{Final results for various $h$ with $k/h = c$ and `order' six. Note the horizontal shifting.}
\includegraphics[width=0.75\textwidth]{plots/6th_order.png}
\end{figure}

\end{document}
