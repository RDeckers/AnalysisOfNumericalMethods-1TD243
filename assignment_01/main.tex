%& -job-name=Assignment1
\input{../documents-common/preamble.tex}
\begin{document}
\maketitle[Assignment 1]{Analysis of Numerical Methods}
\section{}
We discretize the first-order spatial derivative using
\begin{align}
  \pdrv{u}(x_0, t)= D_0\left(I-\frac{h^2}{6}D_+D_-\right)v.\nn
\end{align}
We use $v$ to denote the semi-discrete version of $u$ at an arbitrary gridpoint $x_0$. While $v$ is dependendent upon time we will
not write down the dependence explicitly txo save space. We use the notation $v_\pm$ to indicate the point to the left and
right of an arbitrary but fixed point $v$. Multiple pairs of $+$ or $-$ may be used to indicate multiple steps to the left or right.
\par We know that
\begin{align}
  D_0v = \frac{v_+-v_-}{2h},\nn
  D_+D_-v = \frac{v_+-2v+v_-}{h^2}.
\end{align}
Therefore
\begin{align}
  \pdrv{u}(x_0, t) = \frac{1}{12h}\left[(v_{++}-v_{--})-8(v_+-v_-)\right]
\end{align}

\subsection{}
We have
\begin{align}
  \pdrv[t]v = \frac{1}{12h}\left[-(v_{++}-v_{--})+8(v_+-v_-)\right]
\end{align}
Which in Fourier space gives
\begin{align}
  \pdrv[t]{\hat v} &= \frac{\hat v}{12h}\left[(e^{2h\omega}-e^{-2h\omega})-8(e^{h\omega}-e^{-h\omega})\right],\nn
  &= \frac{i}{6h}\left[-\sin(2h\omega)+8\sin(h\omega)\right]\hat v\nn
  &= \hat P \hat v(t)
\end{align}
We know that wave speed $V_\omega$ is given by
\begin{align}
  V_\omega = \frac{\hat P}{i\omega}
\end{align}
therefore
\begin{align}
 V_\omega = \frac{1}{6h\omega}\left[-\sin(2h\omega)+8\sin(h\omega)\right]
\end{align}
\begin{figure}[!hhp]
\centering
% GNUPLOT: LaTeX picture
\setlength{\unitlength}{0.240900pt}
\ifx\plotpoint\undefined\newsavebox{\plotpoint}\fi
\sbox{\plotpoint}{\rule[-0.200pt]{0.400pt}{0.400pt}}%
\begin{picture}(1500,900)(0,0)
\sbox{\plotpoint}{\rule[-0.200pt]{0.400pt}{0.400pt}}%
\put(110.0,82.0){\rule[-0.200pt]{4.818pt}{0.400pt}}
\put(90,82){\makebox(0,0)[r]{$0$}}
\put(1419.0,82.0){\rule[-0.200pt]{4.818pt}{0.400pt}}
\put(110.0,160.0){\rule[-0.200pt]{4.818pt}{0.400pt}}
\put(90,160){\makebox(0,0)[r]{$0.1$}}
\put(1419.0,160.0){\rule[-0.200pt]{4.818pt}{0.400pt}}
\put(110.0,237.0){\rule[-0.200pt]{4.818pt}{0.400pt}}
\put(90,237){\makebox(0,0)[r]{$0.2$}}
\put(1419.0,237.0){\rule[-0.200pt]{4.818pt}{0.400pt}}
\put(110.0,315.0){\rule[-0.200pt]{4.818pt}{0.400pt}}
\put(90,315){\makebox(0,0)[r]{$0.3$}}
\put(1419.0,315.0){\rule[-0.200pt]{4.818pt}{0.400pt}}
\put(110.0,393.0){\rule[-0.200pt]{4.818pt}{0.400pt}}
\put(90,393){\makebox(0,0)[r]{$0.4$}}
\put(1419.0,393.0){\rule[-0.200pt]{4.818pt}{0.400pt}}
\put(110.0,471.0){\rule[-0.200pt]{4.818pt}{0.400pt}}
\put(90,471){\makebox(0,0)[r]{$0.5$}}
\put(1419.0,471.0){\rule[-0.200pt]{4.818pt}{0.400pt}}
\put(110.0,548.0){\rule[-0.200pt]{4.818pt}{0.400pt}}
\put(90,548){\makebox(0,0)[r]{$0.6$}}
\put(1419.0,548.0){\rule[-0.200pt]{4.818pt}{0.400pt}}
\put(110.0,626.0){\rule[-0.200pt]{4.818pt}{0.400pt}}
\put(90,626){\makebox(0,0)[r]{$0.7$}}
\put(1419.0,626.0){\rule[-0.200pt]{4.818pt}{0.400pt}}
\put(110.0,704.0){\rule[-0.200pt]{4.818pt}{0.400pt}}
\put(90,704){\makebox(0,0)[r]{$0.8$}}
\put(1419.0,704.0){\rule[-0.200pt]{4.818pt}{0.400pt}}
\put(110.0,781.0){\rule[-0.200pt]{4.818pt}{0.400pt}}
\put(90,781){\makebox(0,0)[r]{$0.9$}}
\put(1419.0,781.0){\rule[-0.200pt]{4.818pt}{0.400pt}}
\put(110.0,859.0){\rule[-0.200pt]{4.818pt}{0.400pt}}
\put(90,859){\makebox(0,0)[r]{$1$}}
\put(1419.0,859.0){\rule[-0.200pt]{4.818pt}{0.400pt}}
\put(110.0,82.0){\rule[-0.200pt]{0.400pt}{4.818pt}}
\put(110,41){\makebox(0,0){$0$}}
\put(110.0,839.0){\rule[-0.200pt]{0.400pt}{4.818pt}}
\put(322.0,82.0){\rule[-0.200pt]{0.400pt}{4.818pt}}
\put(322,41){\makebox(0,0){$0.5$}}
\put(322.0,839.0){\rule[-0.200pt]{0.400pt}{4.818pt}}
\put(533.0,82.0){\rule[-0.200pt]{0.400pt}{4.818pt}}
\put(533,41){\makebox(0,0){$1$}}
\put(533.0,839.0){\rule[-0.200pt]{0.400pt}{4.818pt}}
\put(745.0,82.0){\rule[-0.200pt]{0.400pt}{4.818pt}}
\put(745,41){\makebox(0,0){$1.5$}}
\put(745.0,839.0){\rule[-0.200pt]{0.400pt}{4.818pt}}
\put(956.0,82.0){\rule[-0.200pt]{0.400pt}{4.818pt}}
\put(956,41){\makebox(0,0){$2$}}
\put(956.0,839.0){\rule[-0.200pt]{0.400pt}{4.818pt}}
\put(1168.0,82.0){\rule[-0.200pt]{0.400pt}{4.818pt}}
\put(1168,41){\makebox(0,0){$2.5$}}
\put(1168.0,839.0){\rule[-0.200pt]{0.400pt}{4.818pt}}
\put(1379.0,82.0){\rule[-0.200pt]{0.400pt}{4.818pt}}
\put(1379,41){\makebox(0,0){$3$}}
\put(1379.0,839.0){\rule[-0.200pt]{0.400pt}{4.818pt}}
\put(110.0,82.0){\rule[-0.200pt]{0.400pt}{187.179pt}}
\put(110.0,82.0){\rule[-0.200pt]{320.156pt}{0.400pt}}
\put(1439.0,82.0){\rule[-0.200pt]{0.400pt}{187.179pt}}
\put(110.0,859.0){\rule[-0.200pt]{320.156pt}{0.400pt}}
\put(1279,818){\makebox(0,0)[r]{-1/(6*x)*(sin(2*x)-8*sin(x))}}
\put(1299.0,818.0){\rule[-0.200pt]{24.090pt}{0.400pt}}
\put(123,859){\usebox{\plotpoint}}
\put(258,857.67){\rule{3.132pt}{0.400pt}}
\multiput(258.00,858.17)(6.500,-1.000){2}{\rule{1.566pt}{0.400pt}}
\put(123.0,859.0){\rule[-0.200pt]{32.521pt}{0.400pt}}
\put(311,856.67){\rule{3.373pt}{0.400pt}}
\multiput(311.00,857.17)(7.000,-1.000){2}{\rule{1.686pt}{0.400pt}}
\put(271.0,858.0){\rule[-0.200pt]{9.636pt}{0.400pt}}
\put(338,855.67){\rule{3.373pt}{0.400pt}}
\multiput(338.00,856.17)(7.000,-1.000){2}{\rule{1.686pt}{0.400pt}}
\put(325.0,857.0){\rule[-0.200pt]{3.132pt}{0.400pt}}
\put(365,854.67){\rule{3.132pt}{0.400pt}}
\multiput(365.00,855.17)(6.500,-1.000){2}{\rule{1.566pt}{0.400pt}}
\put(378,853.67){\rule{3.373pt}{0.400pt}}
\multiput(378.00,854.17)(7.000,-1.000){2}{\rule{1.686pt}{0.400pt}}
\put(392,852.67){\rule{3.132pt}{0.400pt}}
\multiput(392.00,853.17)(6.500,-1.000){2}{\rule{1.566pt}{0.400pt}}
\put(405,851.67){\rule{3.373pt}{0.400pt}}
\multiput(405.00,852.17)(7.000,-1.000){2}{\rule{1.686pt}{0.400pt}}
\put(419,850.67){\rule{3.132pt}{0.400pt}}
\multiput(419.00,851.17)(6.500,-1.000){2}{\rule{1.566pt}{0.400pt}}
\put(432,849.17){\rule{2.900pt}{0.400pt}}
\multiput(432.00,850.17)(7.981,-2.000){2}{\rule{1.450pt}{0.400pt}}
\put(446,847.67){\rule{3.132pt}{0.400pt}}
\multiput(446.00,848.17)(6.500,-1.000){2}{\rule{1.566pt}{0.400pt}}
\put(459,846.17){\rule{2.700pt}{0.400pt}}
\multiput(459.00,847.17)(7.396,-2.000){2}{\rule{1.350pt}{0.400pt}}
\put(472,844.17){\rule{2.900pt}{0.400pt}}
\multiput(472.00,845.17)(7.981,-2.000){2}{\rule{1.450pt}{0.400pt}}
\put(486,842.17){\rule{2.700pt}{0.400pt}}
\multiput(486.00,843.17)(7.396,-2.000){2}{\rule{1.350pt}{0.400pt}}
\put(499,840.17){\rule{2.900pt}{0.400pt}}
\multiput(499.00,841.17)(7.981,-2.000){2}{\rule{1.450pt}{0.400pt}}
\multiput(513.00,838.95)(2.695,-0.447){3}{\rule{1.833pt}{0.108pt}}
\multiput(513.00,839.17)(9.195,-3.000){2}{\rule{0.917pt}{0.400pt}}
\put(526,835.17){\rule{2.900pt}{0.400pt}}
\multiput(526.00,836.17)(7.981,-2.000){2}{\rule{1.450pt}{0.400pt}}
\multiput(540.00,833.95)(2.695,-0.447){3}{\rule{1.833pt}{0.108pt}}
\multiput(540.00,834.17)(9.195,-3.000){2}{\rule{0.917pt}{0.400pt}}
\multiput(553.00,830.94)(1.797,-0.468){5}{\rule{1.400pt}{0.113pt}}
\multiput(553.00,831.17)(10.094,-4.000){2}{\rule{0.700pt}{0.400pt}}
\multiput(566.00,826.95)(2.918,-0.447){3}{\rule{1.967pt}{0.108pt}}
\multiput(566.00,827.17)(9.918,-3.000){2}{\rule{0.983pt}{0.400pt}}
\multiput(580.00,823.94)(1.797,-0.468){5}{\rule{1.400pt}{0.113pt}}
\multiput(580.00,824.17)(10.094,-4.000){2}{\rule{0.700pt}{0.400pt}}
\multiput(593.00,819.94)(1.943,-0.468){5}{\rule{1.500pt}{0.113pt}}
\multiput(593.00,820.17)(10.887,-4.000){2}{\rule{0.750pt}{0.400pt}}
\multiput(607.00,815.94)(1.797,-0.468){5}{\rule{1.400pt}{0.113pt}}
\multiput(607.00,816.17)(10.094,-4.000){2}{\rule{0.700pt}{0.400pt}}
\multiput(620.00,811.93)(1.489,-0.477){7}{\rule{1.220pt}{0.115pt}}
\multiput(620.00,812.17)(11.468,-5.000){2}{\rule{0.610pt}{0.400pt}}
\multiput(634.00,806.94)(1.797,-0.468){5}{\rule{1.400pt}{0.113pt}}
\multiput(634.00,807.17)(10.094,-4.000){2}{\rule{0.700pt}{0.400pt}}
\multiput(647.00,802.93)(1.123,-0.482){9}{\rule{0.967pt}{0.116pt}}
\multiput(647.00,803.17)(10.994,-6.000){2}{\rule{0.483pt}{0.400pt}}
\multiput(660.00,796.93)(1.489,-0.477){7}{\rule{1.220pt}{0.115pt}}
\multiput(660.00,797.17)(11.468,-5.000){2}{\rule{0.610pt}{0.400pt}}
\multiput(674.00,791.93)(1.123,-0.482){9}{\rule{0.967pt}{0.116pt}}
\multiput(674.00,792.17)(10.994,-6.000){2}{\rule{0.483pt}{0.400pt}}
\multiput(687.00,785.93)(1.214,-0.482){9}{\rule{1.033pt}{0.116pt}}
\multiput(687.00,786.17)(11.855,-6.000){2}{\rule{0.517pt}{0.400pt}}
\multiput(701.00,779.93)(1.123,-0.482){9}{\rule{0.967pt}{0.116pt}}
\multiput(701.00,780.17)(10.994,-6.000){2}{\rule{0.483pt}{0.400pt}}
\multiput(714.00,773.93)(1.026,-0.485){11}{\rule{0.900pt}{0.117pt}}
\multiput(714.00,774.17)(12.132,-7.000){2}{\rule{0.450pt}{0.400pt}}
\multiput(728.00,766.93)(0.950,-0.485){11}{\rule{0.843pt}{0.117pt}}
\multiput(728.00,767.17)(11.251,-7.000){2}{\rule{0.421pt}{0.400pt}}
\multiput(741.00,759.93)(0.824,-0.488){13}{\rule{0.750pt}{0.117pt}}
\multiput(741.00,760.17)(11.443,-8.000){2}{\rule{0.375pt}{0.400pt}}
\multiput(754.00,751.93)(1.026,-0.485){11}{\rule{0.900pt}{0.117pt}}
\multiput(754.00,752.17)(12.132,-7.000){2}{\rule{0.450pt}{0.400pt}}
\multiput(768.00,744.93)(0.728,-0.489){15}{\rule{0.678pt}{0.118pt}}
\multiput(768.00,745.17)(11.593,-9.000){2}{\rule{0.339pt}{0.400pt}}
\multiput(781.00,735.93)(0.890,-0.488){13}{\rule{0.800pt}{0.117pt}}
\multiput(781.00,736.17)(12.340,-8.000){2}{\rule{0.400pt}{0.400pt}}
\multiput(795.00,727.93)(0.728,-0.489){15}{\rule{0.678pt}{0.118pt}}
\multiput(795.00,728.17)(11.593,-9.000){2}{\rule{0.339pt}{0.400pt}}
\multiput(808.00,718.93)(0.728,-0.489){15}{\rule{0.678pt}{0.118pt}}
\multiput(808.00,719.17)(11.593,-9.000){2}{\rule{0.339pt}{0.400pt}}
\multiput(821.00,709.93)(0.786,-0.489){15}{\rule{0.722pt}{0.118pt}}
\multiput(821.00,710.17)(12.501,-9.000){2}{\rule{0.361pt}{0.400pt}}
\multiput(835.00,700.92)(0.652,-0.491){17}{\rule{0.620pt}{0.118pt}}
\multiput(835.00,701.17)(11.713,-10.000){2}{\rule{0.310pt}{0.400pt}}
\multiput(848.00,690.92)(0.704,-0.491){17}{\rule{0.660pt}{0.118pt}}
\multiput(848.00,691.17)(12.630,-10.000){2}{\rule{0.330pt}{0.400pt}}
\multiput(862.00,680.92)(0.590,-0.492){19}{\rule{0.573pt}{0.118pt}}
\multiput(862.00,681.17)(11.811,-11.000){2}{\rule{0.286pt}{0.400pt}}
\multiput(875.00,669.92)(0.704,-0.491){17}{\rule{0.660pt}{0.118pt}}
\multiput(875.00,670.17)(12.630,-10.000){2}{\rule{0.330pt}{0.400pt}}
\multiput(889.00,659.92)(0.590,-0.492){19}{\rule{0.573pt}{0.118pt}}
\multiput(889.00,660.17)(11.811,-11.000){2}{\rule{0.286pt}{0.400pt}}
\multiput(902.00,648.92)(0.539,-0.492){21}{\rule{0.533pt}{0.119pt}}
\multiput(902.00,649.17)(11.893,-12.000){2}{\rule{0.267pt}{0.400pt}}
\multiput(915.00,636.92)(0.637,-0.492){19}{\rule{0.609pt}{0.118pt}}
\multiput(915.00,637.17)(12.736,-11.000){2}{\rule{0.305pt}{0.400pt}}
\multiput(929.00,625.92)(0.539,-0.492){21}{\rule{0.533pt}{0.119pt}}
\multiput(929.00,626.17)(11.893,-12.000){2}{\rule{0.267pt}{0.400pt}}
\multiput(942.00,613.92)(0.536,-0.493){23}{\rule{0.531pt}{0.119pt}}
\multiput(942.00,614.17)(12.898,-13.000){2}{\rule{0.265pt}{0.400pt}}
\multiput(956.00,600.92)(0.539,-0.492){21}{\rule{0.533pt}{0.119pt}}
\multiput(956.00,601.17)(11.893,-12.000){2}{\rule{0.267pt}{0.400pt}}
\multiput(969.00,588.92)(0.536,-0.493){23}{\rule{0.531pt}{0.119pt}}
\multiput(969.00,589.17)(12.898,-13.000){2}{\rule{0.265pt}{0.400pt}}
\multiput(983.00,575.92)(0.497,-0.493){23}{\rule{0.500pt}{0.119pt}}
\multiput(983.00,576.17)(11.962,-13.000){2}{\rule{0.250pt}{0.400pt}}
\multiput(996.00,562.92)(0.497,-0.493){23}{\rule{0.500pt}{0.119pt}}
\multiput(996.00,563.17)(11.962,-13.000){2}{\rule{0.250pt}{0.400pt}}
\multiput(1009.00,549.92)(0.497,-0.494){25}{\rule{0.500pt}{0.119pt}}
\multiput(1009.00,550.17)(12.962,-14.000){2}{\rule{0.250pt}{0.400pt}}
\multiput(1023.58,534.80)(0.493,-0.536){23}{\rule{0.119pt}{0.531pt}}
\multiput(1022.17,535.90)(13.000,-12.898){2}{\rule{0.400pt}{0.265pt}}
\multiput(1036.00,521.92)(0.497,-0.494){25}{\rule{0.500pt}{0.119pt}}
\multiput(1036.00,522.17)(12.962,-14.000){2}{\rule{0.250pt}{0.400pt}}
\multiput(1050.58,506.80)(0.493,-0.536){23}{\rule{0.119pt}{0.531pt}}
\multiput(1049.17,507.90)(13.000,-12.898){2}{\rule{0.400pt}{0.265pt}}
\multiput(1063.00,493.92)(0.497,-0.494){25}{\rule{0.500pt}{0.119pt}}
\multiput(1063.00,494.17)(12.962,-14.000){2}{\rule{0.250pt}{0.400pt}}
\multiput(1077.58,478.67)(0.493,-0.576){23}{\rule{0.119pt}{0.562pt}}
\multiput(1076.17,479.83)(13.000,-13.834){2}{\rule{0.400pt}{0.281pt}}
\multiput(1090.58,463.80)(0.493,-0.536){23}{\rule{0.119pt}{0.531pt}}
\multiput(1089.17,464.90)(13.000,-12.898){2}{\rule{0.400pt}{0.265pt}}
\multiput(1103.58,449.81)(0.494,-0.534){25}{\rule{0.119pt}{0.529pt}}
\multiput(1102.17,450.90)(14.000,-13.903){2}{\rule{0.400pt}{0.264pt}}
\multiput(1117.58,434.67)(0.493,-0.576){23}{\rule{0.119pt}{0.562pt}}
\multiput(1116.17,435.83)(13.000,-13.834){2}{\rule{0.400pt}{0.281pt}}
\multiput(1130.58,419.81)(0.494,-0.534){25}{\rule{0.119pt}{0.529pt}}
\multiput(1129.17,420.90)(14.000,-13.903){2}{\rule{0.400pt}{0.264pt}}
\multiput(1144.58,404.67)(0.493,-0.576){23}{\rule{0.119pt}{0.562pt}}
\multiput(1143.17,405.83)(13.000,-13.834){2}{\rule{0.400pt}{0.281pt}}
\multiput(1157.58,389.69)(0.494,-0.570){25}{\rule{0.119pt}{0.557pt}}
\multiput(1156.17,390.84)(14.000,-14.844){2}{\rule{0.400pt}{0.279pt}}
\multiput(1171.58,373.67)(0.493,-0.576){23}{\rule{0.119pt}{0.562pt}}
\multiput(1170.17,374.83)(13.000,-13.834){2}{\rule{0.400pt}{0.281pt}}
\multiput(1184.58,358.67)(0.493,-0.576){23}{\rule{0.119pt}{0.562pt}}
\multiput(1183.17,359.83)(13.000,-13.834){2}{\rule{0.400pt}{0.281pt}}
\multiput(1197.58,343.69)(0.494,-0.570){25}{\rule{0.119pt}{0.557pt}}
\multiput(1196.17,344.84)(14.000,-14.844){2}{\rule{0.400pt}{0.279pt}}
\multiput(1211.58,327.67)(0.493,-0.576){23}{\rule{0.119pt}{0.562pt}}
\multiput(1210.17,328.83)(13.000,-13.834){2}{\rule{0.400pt}{0.281pt}}
\multiput(1224.58,312.81)(0.494,-0.534){25}{\rule{0.119pt}{0.529pt}}
\multiput(1223.17,313.90)(14.000,-13.903){2}{\rule{0.400pt}{0.264pt}}
\multiput(1238.58,297.54)(0.493,-0.616){23}{\rule{0.119pt}{0.592pt}}
\multiput(1237.17,298.77)(13.000,-14.771){2}{\rule{0.400pt}{0.296pt}}
\multiput(1251.58,281.67)(0.493,-0.576){23}{\rule{0.119pt}{0.562pt}}
\multiput(1250.17,282.83)(13.000,-13.834){2}{\rule{0.400pt}{0.281pt}}
\multiput(1264.58,266.81)(0.494,-0.534){25}{\rule{0.119pt}{0.529pt}}
\multiput(1263.17,267.90)(14.000,-13.903){2}{\rule{0.400pt}{0.264pt}}
\multiput(1278.58,251.67)(0.493,-0.576){23}{\rule{0.119pt}{0.562pt}}
\multiput(1277.17,252.83)(13.000,-13.834){2}{\rule{0.400pt}{0.281pt}}
\multiput(1291.58,236.81)(0.494,-0.534){25}{\rule{0.119pt}{0.529pt}}
\multiput(1290.17,237.90)(14.000,-13.903){2}{\rule{0.400pt}{0.264pt}}
\multiput(1305.58,221.67)(0.493,-0.576){23}{\rule{0.119pt}{0.562pt}}
\multiput(1304.17,222.83)(13.000,-13.834){2}{\rule{0.400pt}{0.281pt}}
\multiput(1318.58,206.81)(0.494,-0.534){25}{\rule{0.119pt}{0.529pt}}
\multiput(1317.17,207.90)(14.000,-13.903){2}{\rule{0.400pt}{0.264pt}}
\multiput(1332.58,191.67)(0.493,-0.576){23}{\rule{0.119pt}{0.562pt}}
\multiput(1331.17,192.83)(13.000,-13.834){2}{\rule{0.400pt}{0.281pt}}
\multiput(1345.58,176.80)(0.493,-0.536){23}{\rule{0.119pt}{0.531pt}}
\multiput(1344.17,177.90)(13.000,-12.898){2}{\rule{0.400pt}{0.265pt}}
\multiput(1358.58,162.81)(0.494,-0.534){25}{\rule{0.119pt}{0.529pt}}
\multiput(1357.17,163.90)(14.000,-13.903){2}{\rule{0.400pt}{0.264pt}}
\multiput(1372.58,147.80)(0.493,-0.536){23}{\rule{0.119pt}{0.531pt}}
\multiput(1371.17,148.90)(13.000,-12.898){2}{\rule{0.400pt}{0.265pt}}
\multiput(1385.00,134.92)(0.497,-0.494){25}{\rule{0.500pt}{0.119pt}}
\multiput(1385.00,135.17)(12.962,-14.000){2}{\rule{0.250pt}{0.400pt}}
\multiput(1399.00,120.92)(0.497,-0.493){23}{\rule{0.500pt}{0.119pt}}
\multiput(1399.00,121.17)(11.962,-13.000){2}{\rule{0.250pt}{0.400pt}}
\multiput(1412.00,107.92)(0.497,-0.494){25}{\rule{0.500pt}{0.119pt}}
\multiput(1412.00,108.17)(12.962,-14.000){2}{\rule{0.250pt}{0.400pt}}
\multiput(1426.00,93.92)(0.497,-0.493){23}{\rule{0.500pt}{0.119pt}}
\multiput(1426.00,94.17)(11.962,-13.000){2}{\rule{0.250pt}{0.400pt}}
\put(352.0,856.0){\rule[-0.200pt]{3.132pt}{0.400pt}}
\put(1279,777){\makebox(0,0)[r]{sin(x)/x}}
\multiput(1299,777)(20.756,0.000){5}{\usebox{\plotpoint}}
\put(1399,777){\usebox{\plotpoint}}
\put(123,859){\usebox{\plotpoint}}
\put(123.00,859.00){\usebox{\plotpoint}}
\put(143.72,858.00){\usebox{\plotpoint}}
\put(164.44,856.97){\usebox{\plotpoint}}
\put(185.07,854.85){\usebox{\plotpoint}}
\put(205.71,852.74){\usebox{\plotpoint}}
\put(226.13,849.04){\usebox{\plotpoint}}
\put(246.56,845.45){\usebox{\plotpoint}}
\put(266.82,840.96){\usebox{\plotpoint}}
\put(287.10,836.52){\usebox{\plotpoint}}
\put(307.14,831.19){\usebox{\plotpoint}}
\put(327.06,825.36){\usebox{\plotpoint}}
\put(346.77,818.87){\usebox{\plotpoint}}
\put(366.50,812.42){\usebox{\plotpoint}}
\put(385.94,805.16){\usebox{\plotpoint}}
\put(405.00,797.00){\usebox{\plotpoint}}
\put(424.35,789.53){\usebox{\plotpoint}}
\put(443.33,781.14){\usebox{\plotpoint}}
\put(462.11,772.32){\usebox{\plotpoint}}
\put(480.75,763.25){\usebox{\plotpoint}}
\put(499.25,753.87){\usebox{\plotpoint}}
\put(517.74,744.45){\usebox{\plotpoint}}
\put(535.88,734.36){\usebox{\plotpoint}}
\put(554.06,724.35){\usebox{\plotpoint}}
\put(571.85,713.66){\usebox{\plotpoint}}
\put(589.68,703.04){\usebox{\plotpoint}}
\put(607.18,691.89){\usebox{\plotpoint}}
\put(624.80,680.91){\usebox{\plotpoint}}
\put(642.36,669.85){\usebox{\plotpoint}}
\put(659.59,658.29){\usebox{\plotpoint}}
\put(676.86,646.80){\usebox{\plotpoint}}
\put(693.70,634.69){\usebox{\plotpoint}}
\put(710.93,623.13){\usebox{\plotpoint}}
\put(727.85,611.11){\usebox{\plotpoint}}
\put(744.43,598.63){\usebox{\plotpoint}}
\put(761.42,586.70){\usebox{\plotpoint}}
\put(778.04,574.28){\usebox{\plotpoint}}
\put(794.85,562.11){\usebox{\plotpoint}}
\put(810.81,548.84){\usebox{\plotpoint}}
\put(827.42,536.41){\usebox{\plotpoint}}
\put(843.74,523.61){\usebox{\plotpoint}}
\put(860.35,511.18){\usebox{\plotpoint}}
\put(876.38,498.02){\usebox{\plotpoint}}
\put(893.00,485.61){\usebox{\plotpoint}}
\put(909.11,472.53){\usebox{\plotpoint}}
\put(925.48,459.77){\usebox{\plotpoint}}
\put(941.42,446.49){\usebox{\plotpoint}}
\put(958.14,434.19){\usebox{\plotpoint}}
\put(974.13,420.97){\usebox{\plotpoint}}
\put(990.23,407.88){\usebox{\plotpoint}}
\put(1006.46,394.95){\usebox{\plotpoint}}
\put(1022.80,382.15){\usebox{\plotpoint}}
\put(1038.73,368.85){\usebox{\plotpoint}}
\put(1055.09,356.08){\usebox{\plotpoint}}
\put(1071.48,343.34){\usebox{\plotpoint}}
\put(1087.48,330.13){\usebox{\plotpoint}}
\put(1103.83,317.35){\usebox{\plotpoint}}
\put(1120.18,304.56){\usebox{\plotpoint}}
\put(1136.58,291.83){\usebox{\plotpoint}}
\put(1152.97,279.10){\usebox{\plotpoint}}
\put(1169.75,266.89){\usebox{\plotpoint}}
\put(1186.15,254.18){\usebox{\plotpoint}}
\put(1202.32,241.20){\usebox{\plotpoint}}
\put(1219.00,228.85){\usebox{\plotpoint}}
\put(1236.15,217.19){\usebox{\plotpoint}}
\put(1252.71,204.68){\usebox{\plotpoint}}
\put(1269.48,192.48){\usebox{\plotpoint}}
\put(1286.42,180.52){\usebox{\plotpoint}}
\put(1303.60,168.90){\usebox{\plotpoint}}
\put(1320.76,157.23){\usebox{\plotpoint}}
\put(1338.08,145.79){\usebox{\plotpoint}}
\put(1355.14,133.98){\usebox{\plotpoint}}
\put(1372.54,122.67){\usebox{\plotpoint}}
\put(1390.16,111.69){\usebox{\plotpoint}}
\put(1407.72,100.63){\usebox{\plotpoint}}
\put(1425.66,90.19){\usebox{\plotpoint}}
\put(1439,82){\usebox{\plotpoint}}
\put(110.0,82.0){\rule[-0.200pt]{0.400pt}{187.179pt}}
\put(110.0,82.0){\rule[-0.200pt]{320.156pt}{0.400pt}}
\put(1439.0,82.0){\rule[-0.200pt]{0.400pt}{187.179pt}}
\put(110.0,859.0){\rule[-0.200pt]{320.156pt}{0.400pt}}
\end{picture}

\input{img/wavespeedRellErr.tex}
\caption{The wavespeed and its relative error as a function of $h\omega$ from $0$ to $\pi$.}
\end{figure}

\subsection{}
The pointwise error is defined as the absolute difference between the exact continous solution and the exact
solution to the (semi-)discrete problem. We know the latter to be $\hat f \exp\left(\hat P t\right)$ from the previous exercise
session, and the former is given to be $f(x-t)$. Thus
\begin{align}
\varepsilon_j(t) &= \left|u(x_,t) - v_{x}(t)\right|\nn
&= \left|f(x-t)-e^{\hat Pt}\hat f\right|\nn
&= \left|e^{i\omega(x-t)}-e^{\frac{i}{6h}\left[-\sin(2h\omega)+8\sin(h\omega)\right]t+i\omega x}\right|\nn%TODO: check \hat f.
&= \left|e^{i\omega x}\right|\left|e^{-i\omega t} - e^{\frac{i}{6h}\left[-\sin(2h\omega)+8\sin(h\omega)\right]t}\right|\nn
&= \left|e^{i\omega x}\right|\left|e^{-i\omega t}\right|\left|1 - e^{it\left(\omega+\frac{1}{6h}\left[-\sin(2h\omega)+8\sin(h\omega)\right]\right)}\right|
\end{align}
Since the first term is the only one which depends on $x$ we can always set it to its maximum value of 1. If we let
\begin{align}
g = i\omega\left(1+\frac{1}{6h\omega}\left[-\sin(2h\omega)+8\sin(h\omega)\right]\right)
\end{align}
We can rewrite $\varepsilon$ as
\begin{align}
\varepsilon_j(t) &= \left|e^{-i\omega t}\right|\left|1-e^{gt}\right|\nn
&\approx \left|e^{-i\omega t}\right|\left|1-1-gt-\O\left(g^2t^2\right)\right|.
\end{align}
We can approximate $g$ using a Taylor-expansion of the $\sin$ terms around $h\omega=0$,
\begin{align}
g &\approx i\omega\left(1+\frac{1}{6h\omega}\left[8h\omega-\frac{8}{6}h^3\omega^3+\frac{8}{120}h^5\omega^5-2h\omega+\frac{8}{6}h^3\omega^3 - \frac{32}{120}h^5\omega^5 + \O\left(h^7\omega^7\right)\right]\right)\nn
&\approx \left(2-\frac{h^4\omega^4}{30}\right)i\omega
\end{align}
therefore
\begin{align}
\varepsilon_j(t) \approx e^{-i\omega t}\left(\frac{h^4\omega^4}{30}-2\right)i\omega t + \O\left(h^7\omega^7+t^2\right)
\end{align}
and the leading order (first order) term of the maximum point-wise error is
\begin{align}
e^{-i\omega t}\left(\frac{h^4\omega^4}{30}-2\right)i\omega t
\end{align}

\section{}
First we discretize the spatial derivative using $D_+D_-$, that is
\begin{align}
\pdrv[t]{\bar{\vec u}} = AD_+D_-\bar{ \vec u}
\end{align}
where $\bar{\vec u}$ is the vector constaining the semidiscrete versions of $u$ and $v$. We now discretize in time as per the Crank-Nicolson scheme,
\begin{align}
 &\bar{\vec u}^+ = \bar{\vec u} + \frac{k}{2}\left[AD_+D_-\bar{ \vec u}+AD_+D_-\bar{ \vec u}^+\right]\nn
 &\to \left(I-\frac{k}{2}AD_+D_-\right)\bar{ \vec u}^+ = \left(I+\frac{k}{2}AD_+D_-\right)\bar{ \vec u}
\end{align}
Where a superscript $\pm$ indicates the next or previous point in time in a similar fashion as a subscript was used to indicate spatial shifts previously.
We know (from lecture notes 6) that the stability requirement for Crank-Nicolson is
\begin{align}
  \left|\frac{1+\frac{k}{2}\lambda}{1-\frac{k}{2}\lambda}\right| \leq 1
\end{align}
where $\lambda$ are the eigenvalues of $AD_+D_-$ in this case. For the above expression to hold, the real parts of these have to be less than or equal to zero.
\par In the fourier domain we can say
\begin{align}
 \pdrv[t]{\hat{\bar{\vec u}}} = AD_+D_- \hat{\bar{\vec u}} = -A\frac{4}{h^2}\sin^2\left(\frac{h\omega}{2}\right)\hat{\bar{\vec u}},
\end{align}
and because a function is stable in the cartesian domain iff it is stable in the fourier domain (the representations are equivalent) the real part of the eigenvalues of $A$ have to be positive.
The eigenvalues of $A$ are trivial to compute and are given by $\lambda_{1,2} = 3,1$, therefore the scheme is stable indepent of $k$ (unconditionally stable)\footnote{Except for the logical bounds on $k$ such as it has to be real and positive.}.
\end{document}
